
\setbeamercovered{transparent}
\newlength{\listLabelLength}

\section{Background}

	\begin{frame}{Motivation and overview}

		\begin{itemize}
			\item<1-> With vast amounts of data, organizations choose to use cloud solutions
			\item<1-> These solutions need to be both efficient and secure
			\item<2-> Recent attacks on access pattern (AP)~\cite{multidimensional-range-queries, inference-attack-islam-14, leakage-abuse-attacks-cash-15, inference-attacks-naveed-15, generic-attacks-kellaris, attacks-tao-of-inference, grubbs-attacks, access-pattern-disclosure, attacks-improved-reconstruction} and communication volume (CV)~\cite{generic-attacks-kellaris, state-of-uniform, attacks-improved-reconstruction, pump-volume-attacks, volume-range-attacks}
			\item<3->
				Existing solutions may be insufficient:
				\begin{itemize}
					\item<1,2,3,6-> \normalsize
						protection against snapshot adversary does not account for AP and CV \\
						\small{CryptDB~\cite{crypt-db}, Arx~\cite{arx}, Seabed~\cite{seabed} and SisoSPIR~\cite{sisospir}}

					\item<1,2,4,6-> \normalsize
						enclaves like SGX are still uncommon and limited in memory \\
						\small{Cipherbase~\cite{cipherbase}, HardIDX~\cite{hardidx}, StealthDB~\cite{stealth-db}, EnclaveDB~\cite{enclave-db}, ObliDB~\cite{oblidb}, Opaque~\cite{opaque} and Oblix~\cite{oblix}}

					\item<1,2,5,6-> \normalsize
						other solutions protect either from one of AP or CV, or use linear scan and full padding \\
						\small{\crypte~\cite{crypte}, Shrinkwrap~\cite{shrinkwrap}, SEAL~\cite{seal} and PINED-RQ~\cite{pined-rq}}
				\end{itemize}
			\item<6-> \epsolute{}: most secure and practical range- and point-query engine in the outsourced database model, that protects both AP and CV using Differential Privacy, while not relying on TEE, linear scan or full padding
		\end{itemize}

	\end{frame}

	\begin{frame}{Cryptographic primitives}

		\begin{columns}[T,onlytextwidth]
			\column{0.475\textwidth}

				\onslide<1->{

					\begin{block}{Symmetric Encryption Scheme}
						\vspace*{1ex}

						\settowidth{\listLabelLength}{\textbf{Key generation}}
						\begin{description}[
							font=\bfseries,
							leftmargin=\dimexpr\listLabelLength+1em\relax,
							labelindent=0pt,
							labelwidth=\listLabelLength%
							]
							\item[Key generation] $k \sample \algo{E.KeyGen}{}$
							\item[Encrypt] $c \sample \algo{E.Enc}{x, k}$
							\item[Decrypt] $x \gets \algo{E.Dec}{c, k}$
						\end{description}

					\end{block}
				}

			\column{0.475\textwidth}

				\onslide<2->{

					\begin{block}{Order-revealing encryption scheme}
						\vspace*{1ex}

						\settowidth{\listLabelLength}{\textbf{Key generation}}
						\begin{description}[
							font=\bfseries,
							leftmargin=\dimexpr\listLabelLength+1em\relax,
							labelindent=0pt,
							labelwidth=\listLabelLength%
						]
							\item[Key generation] $k \sample \algo{ORE.KeyGen}{}$
							\item[Encrypt] $c \sample \algo{ORE.Enc}{x, k}$
							\item[Decrypt] $x \gets \algo{ORE.Dec}{c, k}$
							\item[Compare] $c_1\ \texttt{op}\ c_2 \equiv x_1\ \texttt{op}\ x_2$ \\ $\texttt{op} \in \{ <, \le, =, \ge, > \}$
						\end{description}

					\end{block}
			}

		\end{columns}

		\vspace*{3ex}

		\begin{columns}[T,onlytextwidth]
			\column{0.475\textwidth}
				\onslide<1->{

						For example, AES~\cite{aes-nist} in CBC mode + IV~\cite{modes-of-operation-nist}.
				}

			\column{0.475\textwidth}
				\onslide<2->{

						For example, BCLO~\cite{bclo-ope}, CLWW~\cite{clww-ore}, Lewi-Wu~\cite{lewi-wu-ore}, CLOZ~\cite{cloz-ore} and FH-OPE~\cite{fh-ope}.
				}

		\end{columns}

		\note{
			The note.

		}

	\end{frame}

	\begin{frame}{Access pattern and ORAM}

		\justifying%

		\textbf{Access pattern} is a sequence of memory accesses \oramProgram{}, where each access consists of the memory \emph{location} $o$, read \oramRead{} or write \oramWrite{} \emph{operation} and the \emph{data} $d$ to be written.

		Oblivious RAM (ORAM) is a mechanism that hides the accesses pattern.
		More formally, \oram{} is a protocol between the client \client{} (who accesses) and the server \server{} (who stores), with a guarantee that the view of the server is indistinguishable for any two sequences of the same lengths.

		\begin{columns}[T]
			\column{0.475\textwidth}

				\[
					\begin{split}
						\abs{\oramProgram_1}					& = \abs{\oramProgram_2}							\\
						\textsc{View}_\server (\oramProgram_1)	& \cindist \textsc{View}_\server (\oramProgram_2)
					\end{split}
				\]

			\column{0.475\textwidth}

				\procedure[linenumbering]{\oram{} protocol}{
					\textbf{Client \client}											\>														\> \textbf{Server \server}	\\
					%
					\oramProgram{} = \left. (\oramRead, i, \bot) \right|_{i = 1}^5	\> 														\>							\\
					%
					\text{(client state)}											\> \sendmessageboth*[6em]{\algo{ORAM}{\oramProgram}}	\> \text{(server state)}	\\
					%
					\{ d_1, d_2, d_3, d_4, d_5 \}									\>														\>
				}

		\end{columns}

		\vspace*{1ex}

		For example: Square Root ORAM~\cite{oram-theory}, Hierarchical ORAM~\cite{oram-original}, Binary-Tree ORAM~\cite{binary-tree-oram}, Interleave Buffer Shuffle Square Root ORAM~\cite{shortest-path-oram}, TP-ORAM~\cite{tp-oram}, \textbf{Path-ORAM}~\cite{path-oram} and TaORAM~\cite{taostore}.
		\alert{ORAM incurs at least logarithmic overhead in the number of stored records.~\cite{oram-original}}

	\end{frame}

	\begin{frame}{Privacy}

		\begin{block}{$k$-anonymity~\cite{k-anonymity}}
			\justify%

			Every tuple in the released table must be indistinguishably related to no fewer than $k$ respondents (i.e., similar to at lest $k - 1$ other tuples).

			\begin{itemize}
				\item only with respect to quasi-identifiers
				\item attacks using background knowledge and lack of diversity
				\item a property of a table, not a mechanism (other works with anonymization techniques exist) % l-diversity page 3 bottom
			\end{itemize}
			% https://spdp.di.unimi.it/papers/k-Anonymity.pdf1

		\end{block}

		\pause%

		\begin{block}{$\ell$-diversity~\cite{l-diversity}}
			\justify%

			A block is $\ell$-diverse if it contains at least $\ell$ ``well-represented'' values for the sensitive attribute $S$.
			A table is $\ell$-diverse if every block is $\ell$-diverse.

			Can choose definition of ``well-represented''.
			For example, in \emph{entropy $\ell$-diversity}, every block has at least $\ell$ distinct values for the sensitive attribute.
			In \emph{recursive $\ell$-diversity}, most common value does not appear too often, less common --- not too infrequently.

			% https://personal.utdallas.edu/~muratk/courses/privacy08f_files/ldiversity.pdf

		\end{block}

		% security hides all data completely at a cost, privacy protects information of an individual from inference and de-anonymization

	\end{frame}

	\begin{frame}{Privacy}

		\begin{block}{$t$-closeness~\cite{t-closeness}}
			\justify%

			A block exhibits $t$-closeness if the distance between the distributions of a sensitive attribute in this block and in the whole table is no more than a threshold $t$.
			A table exhibits $t$-closeness if every block does.
			The metric used is the Earth Mover's Distance~\cite{emd}.

			% https://citeseerx.ist.psu.edu/viewdoc/download?doi=10.1.1.157.824&rep=rep1&type=pdf

		\end{block}

		\pause%

		\begin{block}{Differential Privacy, adapted from~\cite{our-data-ourselves, differential-privacy-original}}
			\justify%

			A randomized algorithm \algo{A} is $(\epsilon, \delta)$-differentially private if for all $\database_1 \sim \database_2 \in \searchKeyDomain^\dataSize$, and for all subsets $\mathcal{O}$ of the output space of \algo{A},
			\[
				\probability{ \algo{A}{ \database_1 } \in \mathcal{O} } \leq \exp(\epsilon) \cdot \probability{ \algo{A}{ \database_2 } \in \mathcal{O} } + \delta \; .
			\]

			\begin{itemize}
				\item Laplace Perturbation Algorithm (LPA)~\cite[Theorem 1]{differential-privacy-original}
				\item Differentially Private Sanitizer
				\item Composition Theorem (disjoint and non-disjoint sets)
			\end{itemize}

		\end{block}

	\end{frame}

	\begin{frame}{Encalves, SGX and ZeroTrace}

		\begin{block}{Software Guard Extensions (SGX)~\cite{sgx-1, sgx-2, sgx-3, sgx-manual, sgx-explained}}
			\justify%

			\vspace*{1ex}

			\begin{columns}[T,onlytextwidth]
				\column{0.6\textwidth}

					Features:
					\begin{itemize}
						\item Set of new x86 instructions
						\item Virtual isolation within ``enclaves''
						\item The entire non-enclave stack is untrusted
						\item Can swap/re-encrypt pages from RAM
						\item Application declares enclave and non-enclave parts
						\item Enclave should manipulate sensitive data, e.g., keys
					\end{itemize}

				\column{0.4\textwidth}

					Issues:
					\begin{itemize}
						\item Small $\approx \SI{96}{\mega\byte}$ of ``trusted'' memory
						\item Enclave code is significantly slower
						\item No direct I/O or syscalls
						\item \alert{Leaks access pattern}
					\end{itemize}

			\end{columns}

		\end{block}

		\vspace*{1ex}

		\begin{block}{ZeroTrace~\cite{zerotrace}}
			\justify%

			PathORAM~\cite{path-oram} or CircuitORAM~\cite{circuit-oram} in SGX, given that the enclave code leaks access pattern. Uses oblivious operations.

		\end{block}

	\end{frame}


