\section{Model}

	\begin{frame}{Snapshot adversary / Data-at-rest}

		\begin{columns}[T,onlytextwidth]
			\column{0.5\textwidth}

				\begin{block}{Adversary steals the hard drive}

					\begin{itemize}
						\item<1-> Cannot query fully encrypted blob \\ \small{cannot outsource key}
						\item<1-> Download-decrypt-query is inefficient
						\item<1-> Relaxing from absolute (semantic) security
						\item<2-> Searchable symmetric encryption (SSE)~\cite{sse-improved}
						\item<3-> Fully-homomorphic encryption (FHE)~\cite{fhe}
						\item<4-> Functional Encryption~\cite{functional-encryption}
						\item<5-> Property-preserving encryption (PPE)~\cite{ope-original, ore-original}
					\end{itemize}

				\end{block}

			\column{0.5\textwidth}

				\onslide<6->{

					\begin{block}{Attacks}

						\begin{itemize}
							\item Usually require auxillary knowledge \\ \small{e.g., distribution}
							\item \normalsize Not necessarily ``full'' reconstruction
							\item \normalsize Lots of attacks~\cite{inference-attacks-naveed-15, attacks-tao-of-inference, grubbs-attacks, knn-attacks}
						\end{itemize}

					\end{block}

				}

		\end{columns}

	\end{frame}

	\begin{frame}{Persistent adversary / AP and CV leakage}

		\begin{columns}[T,onlytextwidth]
			\column{0.5\textwidth}

			\onslide<1->{
				\begin{block}{Access Pattern}

					\begin{itemize}
						\item Which query ``touches'' which records
						\item Applicable to all types of queries
						\item Usually mitigated with ORAM
						\item Attacks~\cite{multidimensional-range-queries, inference-attack-islam-14, leakage-abuse-attacks-cash-15, inference-attacks-naveed-15, generic-attacks-kellaris, attacks-tao-of-inference, grubbs-attacks, access-pattern-disclosure, attacks-improved-reconstruction}
					\end{itemize}

				\end{block}
			}

			\column{0.5\textwidth}

				\onslide<2->{

					\begin{block}{Communication Volume}

						\begin{itemize}
							\item The size of the answer (in bytes or records)
							\item More often applicable to range queries
							\item Usually mitigated with padding / noise
							\item Attacks~\cite{generic-attacks-kellaris, state-of-uniform, attacks-improved-reconstruction, pump-volume-attacks, volume-range-attacks}
						\end{itemize}

					\end{block}

				}

		\end{columns}

		\vspace*{2ex}

		\onslide<3->{
			Can we put forth a definition that would imply protection against all these attacks?
		}

	\end{frame}

	\begin{frame}{Differentially Private Outsourced Database System}

		\begin{definition}[\alert{Computationally Differentially Private Outsourced Database System (CDP-ODB)}]
			\justify%

			We say that an outsourced database system \protocol{} is $(\epsilon, \delta)$-computationally differentially private (a.k.a.~CDP-ODB) if for every polynomial time distinguishing adversary \adversary{}, for every neighboring databases $\database \sim \database^\prime$, and for every query sequence $\fromNtoM{\query}{1}{m} \in \querySet^m$ where $m = \mathsf{poly}(\lambda)$,

			\begin{multline*}
				\probability{\adversary \left( 1^\lambda, \view{\protocol, \server}{\database, \fromNtoM{\query}{1}{m}} \right) = 1 } \leq \\
				\exp{\epsilon} \cdot \probability{\adversary \left( 1^\lambda, \view{\protocol ,\server}{\database^\prime, \fromNtoM{\query}{1}{m}} \right) = 1} + \delta + \negl \; ,
			\end{multline*}
			the probability is over the randomness of the distinguishing adversary \adversary{} and the protocol \protocol{}.
		\end{definition}

		\pause%

		Note:
		\begin{itemize}
			\item Entire view of the adversary is DP-protected
			\item Implies protection against communication volume and access pattern leakages
			\item Query sequence $\fromNtoM{\query}{1}{m} \in \querySet^m$ is fixed
			\item $\negl$ accounts for the computational (as opposed to theoretical) DP definition
		\end{itemize}

	\end{frame}
