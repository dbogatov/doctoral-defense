% cSpell:disable
% chktex-file 15

% CLI arguments

\usepackage{ifthen}
\ifthenelse%
	{\equal{\generatenotes}{hide}}
	{\newcommand{\notesOption}{hide notes}}
	{
		\ifthenelse%
			{\equal{\generatenotes}{second}}
			{\newcommand{\notesOption}{show notes on second screen}}
			{\newcommand{\notesOption}{show only notes}}
	}%

% packages

\usepackage[
	orientation=landscape,
	size=custom,
	width=24,
	height=14,
	scale=0.6
]{beamerposter}

\usepackage[
	lambda,
	advantage,
	operators,
	sets,
	landau,
	probability,
	notions,
	logic,
	ff,
	mm,
	primitives,
	events,
	complexity,
	asymptotics,
	keys
]{cryptocode}
\usepackage{pgfpages}
\usepackage{booktabs}
\usepackage{bm}
\usepackage{mathtools}
\usepackage{listings}
\usepackage{fancybox}
\usepackage{bm}
\usepackage{multicol}
\usepackage{xpatch}
\usepackage{hyperref}
\usepackage{hyperxmp}
\usepackage{multirow}
\usepackage{xparse}
\usepackage{hyphenat}
\usepackage{xfrac}
\usepackage{caption}
\usepackage{enumitem}

\setitemize{label=	\usebeamerfont*{itemize item}
					\usebeamercolor[fg]{itemize item}
					\usebeamertemplate{itemize item}}

\usepackage{siunitx}
\sisetup{range-phrase=--}

\usepackage[
	backend=bibtex,
	style=numeric,
	sorting=ynt
]{biblatex}
\addbibresource{bibfile.bib}

\usepackage{background}
\backgroundsetup{
	placement=center,
	scale=3,
	contents={\begin{minipage}{1.0\textwidth}\centering\wm\end{minipage}},
	opacity=0.02,
	color=gray,
	angle=0
}
\setbeamertemplate{background}{\BgMaterial}

\usepackage{eucal}

% settings

\graphicspath{{./graphics/}}

\makeatletter
\defbeameroption{show only notes}[]%
{
	\beamer@notestrue%
	\beamer@notesnormalsfalse%
}
\makeatother

\setbeameroption{\notesOption}

\setsansfont{CMU Serif}
\setmonofont{CMU Typewriter Text}

\makeatletter
\let\@@magyar@captionfix\relax % chktex 21
\makeatother

\logo{
	\includegraphics[
		width=1cm,
		keepaspectratio
	]{logo}\hspace{\dimexpr\paperwidth-1.5cm}\vspace{-30pt}
}

\titlegraphic{%
	\begin{picture}(0,0)
		\put(620,-180){\makebox(0,0)[rt]{\includegraphics[width=5cm]{logo}}}
	\end{picture}
}

\makeatletter
	\def\beamer@framenotesbegin{% at beginning of slide
		\usebeamercolor[fg]{normal text}
			\gdef\beamer@noteitems{}%
			\gdef\beamer@notes{}%
	}
\makeatother

\definecolor{Darker}{HTML}{2C0000}
\definecolor{Lighter}{HTML}{CC0000}
\definecolor{Text}{RGB}{11,32,76}

\hypersetup{
	colorlinks=true,
	linkcolor=Darker,
	urlcolor=Darker,
	citecolor=Lighter,
	pdfpagemode=FullScreen,
	pdfdisplaydoctitle=true,
	pdfmenubar=false,
	pdfpagelayout=SinglePage
}

\captionsetup[figure]{labelformat=empty}

\makeatletter
	\setlength{\metropolis@progressinheadfoot@linewidth}{2pt}
\makeatother

\setbeamercolor{background canvas}{bg=white}
\setbeamercolor{normal text}{fg=Text}
% \setbeamercolor{normal text}{fg=Darker}
\setbeamercolor{frametitle}{bg=Darker, fg=white}
\setbeamercolor{alerted text}{fg=Lighter}
\setbeamercolor{example text}{fg=Darker}
{
	\usebeamercolor[fg]{alerted text}
	\usebeamercolor[fg]{example text}
	\usebeamercolor[fg]{normal text}
}

% definitions

\xpatchbibmacro{name:andothers}{%
	\bibstring{andothers}%
}{%
	\bibstring[\emph]{andothers}%
}{}{}

\makeatletter
	\newcommand{\manuallabel}[2]{\def\@currentlabel{#2}\label{#1}} % chktex 21
\makeatother

\newenvironment<>{fixblock}[1]{%
	\begin{block}{#1}
		\vspace{0pt}
		#2
}{
	\end{block}
}

\newenvironment{fixnote}{\startfixnote}{}
\def\startfixnote#1\end{\note{#1}\end} % chktex 9 chktex 14

% Add bidirectional arrow to cryptocode
\makeatletter
	\newcommandx*{\sendmessageboth}[2][1=<->]{%
		\sendmessage{#1}{#2}%
	}
	\WithSuffix\newcommand\sendmessageboth*[2][\pcdefaultmessagelength]{%
		\begingroup%
			\renewcommand{\@pcsendmessagetop}{\let\halign\@pc@halign$\begin{aligned}#2\end{aligned}$}% chktex 21
			\sendmessage{<->}{length=#1}%
		\endgroup%
	}
	\renewcommand{\pccomment}[1]{{\rhd\;\text{\scriptsize#1}}} % chktex 21
\makeatother

\newcommand{\epsolute}{\ensuremath{\mathcal{E}}psolute}
\DeclareDocumentCommand{\algo}{ m g }{%
	{%
		\textsc{#1}%
		\IfNoValueF{#2}{\ensuremath{\left( #2 \right)}}%
	}%
}
\newcommand{\client}{\ensuremath{\mathcal{C}}}
\newcommand{\user}{\ensuremath{\mathcal{U}}}
\newcommand{\server}{\ensuremath{\mathcal{S}}}
\newcommand{\oram}{\ensuremath{\textsc{ORAM}}}
\newcommand{\oramProgram}{\ensuremath{\mathbf{y}}}
\newcommand{\oramRead}{\ensuremath{\mathbf{r}}}
\newcommand{\oramWrite}{\ensuremath{\mathbf{w}}}
